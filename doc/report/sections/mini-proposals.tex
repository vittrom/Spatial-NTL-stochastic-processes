% !TEX root = ../main.tex

% Mini-proposals section

\section{Mini-proposals}  \label{proposals}

% each mini-proposal gets its own subsection
\subsection{Proposal 1: NTL species sampling and generalized Chinese restaurant process} % enter your proposal title

The work of \cite{james2006poisson}, that guided our analysis, was motivated by the observation that despite the numerous applications suitable for NTR priors, NTR processes were not being used as extensively as they could have been. The Dirichlet process constituted the only exception and it is still ubiquitously used in Bayesian nonparametrics applications. 

\cite{james2006poisson} identified several reasons for the absence of NTR priors from the Bayesian nonparametrics literature and addressed them in his work.

Firstly, he noticed that the NTR process of \cite{doksum1974tailfree} was defined only on the real line, $\bbR$, therefore restricting its application to processes in the real space. Hence, NTR processes could not be applied to situations where inference on more complex spaces was required. To address this limitation, \cite{james2006poisson} proposed the spatial neutral-to-the-right process (SPNTR), an extension of NTR processes to arbitrary Polish spaces. From SPNTR, \cite{james2006poisson} derived a new class of random probability measures called NTR species sampling models. This work focuses on characterizing SPNTL processes similarly to \cite{james2006poisson}, however, the possibility to derive analogous NTL species sampling models is not explored and could be further investigated following the analysis of \cite{james2006poisson}.

Secondly, and probably the most important problem addressed in \cite{james2006poisson}, is that NTR processes are, in general, not tractable and there is no simple way to sample from them, limiting their practical application. On the other hand, Dirichlet processes can easily be sampled through the Chinese restaurant process. \cite{james2006poisson} derives a generalized Chinese restaurant process sampling scheme for SPNTR processes, making their implementation more approachable. Our work has not developed a similar sampling scheme for SPNTL.

We believe that exploring the two research directions presented in this proposal is essential to make SPNTL processes available for practical use. The analysis could be performed following the work of \cite{james2006poisson} and can bring forward the theoretical development of NTL processes.

% each mini-proposal gets its own subsection
\subsection{Proposal 2: Application to streaming data} % enter your proposal title
As mentioned in \Cref{summary}, the approach of transforming an NTL into an NTR process, performing the analysis as if it were an NTR process and transforming the NTR back to NTL requires knowledge of the full dataset and cannot be applied to streaming data. SPNTL allow to overcome this issue providing a way to model data directly as a NTL process. In this paper we focused on deriving the theoretical results for SPNTL, however, it is interesting to observe the advantage obtained by directly modeling the NTL process. 

Furthermore, it would be interesting to use SPNTL to model streaming data and compare inference performance to established methods. In particular, we think that by being able to model not only the temporal component of the process but also the action on more complex spaces SPNTL could lead to improved performance.

Testing performance of SPNTL models could highlight practical weaknesses of the models and foster ideas for new research directions and theoretical developments.

Data that could be used for this project could come from the recommender systems literature, where they could perhaps be used to address the cold start problem, a well-known problem for recommender systems. Another application could be in the context of random graphs \cite{bloem2017preferential,bloem2018sampling}. 
% ...