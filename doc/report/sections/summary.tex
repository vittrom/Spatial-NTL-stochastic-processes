% !TEX root = ../main.tex

% Summary section

\section{Summary} \label{summary}
A sequence of random variables $X = (X_1, \ldots, X_k) \in \bbR^k_+$ is said to be neutral-to-the-left (NTL) if the increments 
\begin{equation*}
R_j = \frac{X_j}{\sum_{i=1}^j X_i}, \quad j=1, \ldots, k
\end{equation*} 
form a sequence of mutually independent random variables in $[0, 1]$ \cite{bloem2018sampling}. Such processes are common in statistics and can be used to describe, for instance, the generating process of left censored data or the asymptotic degree distribution in preferential attachment random graphs \cite{bloem2017preferential}. In the statistics and probability literature and, especially, in Bayesian nonparametrics, NTL processes have not received as much attention as their analogous counterpart, neutral-to-the-right (NTR) processes \footnote{Reversing an NTL process gives an NTR process.}. The disparity in the development of theory for NTL and NTR processes can be attributed to two factors:

\begin{enumerate}
	\item NTR processes are more common than NTL processes, therefore more focus has been put on the development of theory for the former.
	\item Since NTL and NTR are symmetric opposites \cite{bloem2018sampling}, an NTL process can be transformed into an NTR for the analysis and back to an NTL after the analysis.
\end{enumerate}

However, the increased attention received by the study of random graphs as well as the interest to model streaming data has made NTL processes more popular and highlighted the need to develop a theory for NTL processes. In fact, the approach used so far in the study of NTL processes cannot be applied to situations where data is collected in a streaming fashion. Transforming and NTL into an NTR process requires knowledge of the full dataset hence, analysis of NTL processes of streaming data cannot be performed with the current approach.

To overcome this issue and expand existing literature on NTL processes, we present here the spatial neutral-to-the-left (SPNTL) stochastic process. SPNTL enables Bayesian modeling of NTL data on complex spaces and can be used as prior in Bayesian nonparametric models. In this paper, following the theoretical analysis of \cite{james2006poisson} for spatial neutral-to-the-right (SPNTR) processes, we derive a characterization of the posterior distribution of SPNTL using results from Poisson partition calculus \cite{james2005poisson}.

While the results in our paper are important to develop a theory for NTL processes, our work does not cover all theoretical contributions of \cite{james2006poisson} for the setting of NTL processes. Deriving additional results for SPNTL is left as future work and discussed more in detail in \Cref{proposals} together with possible applications. 